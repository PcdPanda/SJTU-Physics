\documentclass[a4paper]{article}
\usepackage{amsmath}
\usepackage{amssymb}
\usepackage{braket}%量子力学符号
\usepackage{geometry}
\usepackage{enumerate}
\usepackage{natbib}
\usepackage{float}%稳定图片位置
\usepackage{graphicx,subfig}%画图
\usepackage{caption}
\usepackage[english]{babel}
\usepackage{indentfirst}%缩进
\usepackage{enumerate}%加序号
\usepackage{multirow}%合并行
\usepackage{hyperref}
\hypersetup{hypertex=true, colorlinks=true, linkcolor=black, anchorcolor=black, citecolor=black}
\title{\Large \textbf{VP390 Review 3}\\
\author{\textbf{Pan, Chongdan ID:516370910121}\\
}
}
\begin{document}
\maketitle
\begin{enumerate}
    \item What kind of experiment disproved the Thomson’s model of the atom? The Rutherford’s model that correctly reproduced the experimental data assumed that 
    \\ \textbf{Some $\alpha$ particles will be scattered after hitting an atom. It assumes that the positive charge gathers together at the center of atom}
    \item According to Bohr’s model of the hydrogen atom, the electron moves in a circular orbit such that the magnitude of its angular momentum is
    \\ \textbf{n$\hbar$}
    \item According to Bohr’s model of the hydrogen atom, the electron moving in a circular orbit with $L_z=n\hbar$ emits electromagnetic radiation.
    \\ \textbf{False}
    \item State the postulates of Bohr’s model of the hydrogen atom
    \\ \textbf{The orbits with different energy level is discrete. When the state or the orbits of one electron change, the atom will radiate electromagnetic wave.}
    \item According to Bohr’s model, when is the hydrogen atom able to emit or absorb electromagnetic
    radiation? What is its frequency equal to?
    \\ \textbf{When the electron changes its orbit. $\nu=\frac{Z^2(E_2-E_1)}{h}$}
    \item The energy levels of the hydrogen atom are $E_n=\cdots E_0$, where $E_0=13.6$eV 
    \\ \textbf{$E_n=\frac{E_0}{(n+1)^2}$}
    \item In quantum mechanics, the state of a system with the lowest energy is called the $\cdots$ state
    \\ \textbf{ground state}
    \item The radius of the first Bohr orbit in the hydrogen atom is of the order of
    \\ \textbf{$10^{-11}$}
    \item State the correspondence principle
    \\ \textbf{When the quantum is large, the result is just like classical result}
    \item What was the idea of the Frank-Hertz experiment and what did it confirm?
    \\ \textbf{Hit atom with electron with increasing energy. The atom only radiate when the electron's energy is above certain value. The multiple times of the value will cause multiple radiation.}
    \item In the original Frank-Hertz experiment with the gas tube filled with mercury atoms, dips in the current were observed for the values of the accelerating potential differing by ca. 4.9 V. Does it mean that energy levels in the mercury atom are equally spaced?
    \\ \textbf{No, it just cause twice radiation}
    \item In the original Frank-Hertz experiment with the tube filled with mercury atoms, dips in the current were observed for the values of the accelerating potential differing by ca. 4.9 V. How would you interpret the second dip that is observed at ca. 9.8 V?
    \\ \textbf{Two same radiation was caused}
    \item The ideal blackbody is modeled as a cavity that $\cdots$ absorbs incident radiation
    \\ \textbf{completely}
    \item What is the UV catastrophe?
    \\ \textbf{When $\lambda$ is small, $u(\lambda)\rightarrow\infty$}
    \item State the Stefan-Boltzman law and Wien law. Explain all symbols in the formulas
    \\ \textbf{$\frac{P}{A}=\sigma T^4,\lambda_mT=$const}
    \item The average energy per resonant mode of electromagnetic field in the cavity modeling the ideal
    black body is $\cdots$ in the classicalal approach and $\cdots$ in Planck’s model.
    \\ \textbf{$kT,\frac{h\nu}{e^{h\nu/kT}-1}$}
    \item Where does quantum mechanics appear in the Planck’s model of blackbody radiation?
    \\ \textbf{Energy of electromagnetic radiation is quantized}
    \item How do waves with small wavelengths contribute to energy radiated from a unit volume of the ideal blackbody?
    \\ \textbf{When wavelength is very small, energy approaches 0, otherwise it follows the classical result, which decreases with larger wavelength}
    \item Write down the photoelectric effect equation. Explains all symbols
    \\ \textbf{$h\nu=\phi+K_{max}$}
    \item Assuming that the magnitude of the electron’s charge is known, how can we use the data from a photoelectric effect measurement to find the value of the Planck’s constant?
    \\ \textbf{Draw the relation between $\nu$ and $K_{max}$, the slope is $\frac{h}{e}$}
    \item In the model of Compton scattering, light is treated as $\cdots$ scattered off resting $\cdots$
    \\ \textbf{Particles, electrons}
    \item What does the Young’s double-slit experiment illustrate?
    \\ \textbf{Light has the property of waves to cause inference}
\end{enumerate}
\end{document}