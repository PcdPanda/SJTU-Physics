\documentclass[a4paper]{article}
\usepackage{amsmath}
\usepackage{amssymb}
\usepackage{braket}%量子力学符号
\usepackage{geometry}
\usepackage{enumerate}
\usepackage{natbib}
\usepackage{float}%稳定图片位置
\usepackage{graphicx,subfig}%画图
\usepackage{caption}
\usepackage[english]{babel}
\usepackage{indentfirst}%缩进
\usepackage{enumerate}%加序号
\usepackage{multirow}%合并行
\usepackage{hyperref}
\usepackage{verbatim}
\hypersetup{hypertex=true, colorlinks=true, linkcolor=black, anchorcolor=black, citecolor=black}
\title{\Large \textbf{VP390 Problem Set 6}\\
\author{\textbf{Pan, Chongdan ID:516370910121}\\
}
}
\begin{document}
\maketitle
\section{Problem 1}
    For this problem, I choose the parameters of Gauss Wave Packet as: 
    \\$t0=1,\alpha=10^{-34},m=9\times10^{-31},p_0=10^{30}$
    \\Then we get graphs when $t=0,1,2,3$ as:
    \begin{figure}[H]
        \centering
        \includegraphics[scale=0.75]{P1-0.png}
        \caption{$t=0$}
    \end{figure}
    \begin{figure}[H]
        \centering
        \includegraphics[scale=0.75]{P1-1.png}
        \caption{$t=1$}
    \end{figure}
    \begin{figure}[H]
        \centering
        \includegraphics[scale=0.75]{P1-2.png}
        \caption{$t=2$}
    \end{figure}
    \begin{figure}[H]
        \centering
        \includegraphics[scale=0.75]{P1-3.png}
        \caption{$t=3$}
    \end{figure}
\section{Problem 2}
\begin{enumerate}[(a)]
    \item For the third maximum point, we get $E=V_0+\frac{\hbar^2\pi^2}{8ma^2}3^2$
    \\$100\times1.6\times10^{-19}=9\frac{(1.054\times10^{-34})^2\pi^2}{8\times9.1\times10^{-31}\times(10^{-10})^2}+V_0$
    \\$V_0=15.28$eV
    \item For $E\rightarrow V_0, T\rightarrow\frac{1}{1+\frac{2ma^2V_0}{\hbar^2}}$
    \\Then $a=1.0626\times10^{-9}$m, and the width is $2.13\times10^{-9}m$
\end{enumerate}
\section{Problem 3}
For the barrier problem we get three equations:
\\ \[\left\{\begin{array}{ll}
    \psi_1(x)=Ae^{ik_1x}+Be^{-ik_1x}&x\leq -a\\
    \psi_2(x)=Ce^{ik_2x}+De^{-ik_2x}&|x|<a\\ 
    \psi_3(x)=Fe^{ik_1x}&x\geq a\\  
    \end{array}\right.\]
where $k_1=\frac{\sqrt{2mE}}{\hbar}$ and $k_2=\frac{\sqrt{2m(E-V_0)}}{\hbar}$
\\Since $\psi(x)$ and $\psi'(x)$ are continuous at $x=\pm a$, we get:
\\ \[\left\{\begin{array}{ll}
    Ae^{-ik_1a}+Be^{ik_1a}=Ce^{-ik_2a}+De^{ik_2a}&(1)\\
    Ce^{ik_2a}+De^{-ik_2a}=Fe^{ik_1a}&(2)\\
    k_1(Ae^{-ik_1a}-Be^{ik_1a})=k_2(Ce^{-ik_2a}-De^{ik_2a})&(3)\\
    k_2(Ce^{ik_2a}-De^{-ik_2a})=k_1Fe^{ik_1a}&(4)
    \end{array}\right.\]
\\$k_2*(2)-(4)$ we get $D=\frac{k_2-k_1}{2k_2}Fe^{i(k_1+k_2)a}$
\\$k_2*(2)+(4)$ we get $C=\frac{k_1+k_2}{2k_2}Fe^{i(k_1-k_2)a}$
\\$k_1*(1)+(3)$ we get $2k_1A=(k_1+k_2)Ce^{i(k_1-k_2)a}+(k_1-k_2)De^{i(k_1+k_2)a}$
\\Then $\frac{A}{F}=\frac{1}{4k_1k_2}[(k_1+k_2)^2e^{i(k_1-2k_2)a}+(k_1-k_2)^2e^{i(k_1+2k_2)a}]$
\\$\frac{F}{A}=\frac{4k_1k_2}{(k_1+k_2)^2e^{i(k_1-2k_2)a}+(k_1-k_2)^2e^{i(k_1+2k_2)a}}=\frac{4k_1k_2e^{i(-k_1+2k_2)a}}{(k_1+k_2)^2+(k_1-k_2)^2e^{4ik_2a}}$
\\\\$k_1*(1)-(3)$ we get $2k_1B=(k_1-k_2)Ce^{-i(k_1+k_2)a}+(k_1+k_2)De^{i(k_2-k_1)a}$
\\$\frac{B}{F}=\frac{1}{4k_1k_2}[(k_1^2-k_2^2)e^{-i2k_2a}+(k_2^2-k_1^2)e^{i2k_2a}]$
\\$\frac{B}{A}=\frac{(k_1^2-k_2^2)e^{-i2k_2a}+(k_2^2-k_1^2)e^{i2k_2a}}{(k_1+k_2)^2e^{i(k_1-2k_2)a}+(k_1-k_2)^2e^{i(k_1+2k_2)a}}=\frac{(k_1^2-k_2^2)e^{-ik_1a}+(k_2^2-k_1^2)e^{i(-k_1+4k_2a)}}{(k_1+k_2)^2+(k_1-k_2)^2e^{4ik_2a}}$
\\Hence $T+R=(\frac{F}{A})^2+(\frac{B}{A})^2=1$
\section{Problem 4}
For the barrier problem we get three equations:
\\ \[\left\{\begin{array}{ll}
    \psi_1(x)=Ae^{ik_1x}+Be^{-ik_1x}&x\leq -a\\
    \psi_2(x)=Ce^{ik_2x}+De^{-ik_2x}&|x|<a\\ 
    \psi_3(x)=Fe^{ik_1x}&x\geq a\\  
    \end{array}\right.\]
where $k_1=\frac{\sqrt{2mE}}{\hbar}$ and $k_2=\frac{\sqrt{2m(E+V_0)}}{\hbar}$
\\\\We can plug them into the solution of rectangular barrier since the equations are in same forms by replacing $V_0$ with $-V_0$
\\Then we get $\frac{1}{T}=1+\frac{1}{4}\frac{V_0^2}{E(E+V_0)}\sin^2(2\sqrt{\frac{2ma^2}{\hbar^2}(E+V_0)})$
\\Hence we get the result $\frac{1}{T}=1+\frac{1}{4}\frac{V_0^2}{E(E+V_0)}\sin^2(2k_2a)$
\\\\For perfect transmission, we need $2k_2a=2\sqrt{\frac{2ma^2}{\hbar^2}(E+V_0)}=n\pi$
\\Then $E=\frac{\pi^2\hbar^2}{8ma^2}n^2-V_0$
\section{Problem 5}
\noindent $E=V_0+\frac{\hbar^2\pi^2}{8ma^2}n^2$
\\$a=0.5\times10^{-10},n=1,E=0.9\times1.6\times10^{-19}$
\\$V_0=-5.885\times10^{-18}$J
\section{Problem 6}
\begin{enumerate}[(a)]
    \item $-[\hat{B},\hat{A}]=-\hat{B}\hat{A}+\hat{A}\hat{B}=\hat{A}\hat{B}-\hat{B}\hat{A}=[\hat{A},\hat{B}]$
    \item $\hat{A}[\hat{B},\hat{C}]+[\hat{A},\hat{C}]\hat{B}=\hat{A}\hat{B}\hat{C}-\hat{A}\hat{C}\hat{B}+\hat{A}\hat{C}\hat{B}-\hat{C}\hat{A}\hat{B}=(\hat{A}\hat{B})\hat{C}-\hat{C}(\hat{A}\hat{B})=[\hat{A}\hat{B},\hat{C}]$
    \item $[\hat{A},\hat{C}]+[\hat{B},\hat{C}]=\hat{A}\hat{C}-\hat{C}\hat{A}+\hat{B}\hat{C}-\hat{C}\hat{B}=(\hat{A}+\hat{B})\hat{C}-\hat{C}(\hat{A}+\hat{B})=[\hat{A}+\hat{B},\hat{C}]$
    \item $[\hat{A},\gamma]=\hat{A}\gamma-\gamma\hat{A}=\gamma\hat{A}-\gamma\hat{A}=0$
    \item $[\hat{A}-\alpha,\hat{B}-\beta]=(\hat{A}-\alpha)(\hat{B}-\beta)-(\hat{B}-\beta)(\hat{A}-\alpha)=\hat{A}\hat{B}-\hat{B}\hat{A}-\alpha(\hat{B}-\hat{B})-\beta(\hat{A}-\hat{A})=[\hat{A},\hat{B}]$
    \item $[\hat{A},f(\hat{A})]=\hat{A}f(\hat{A})-f(\hat{A})\hat{A}$
    \\Since $f(\hat{A})$ is an analytic function, it can be expressed in the form $f(\hat{A})=\sum_{i=0}^{\infty}k_i \hat{A}^i$
    \\Then $\hat{A}f(\hat{A})-f(\hat{A})\hat{A}=\sum_{i=0}^{\infty}k_i \hat{A}^{i+1}-\sum_{i=0}^{\infty}k_i \hat{A}^{i+1}=0$
    \\Hence $[\hat{A},f(\hat{A})]=0$
\end{enumerate}
\section{Problem 7}
\begin{enumerate}[(a)]
    \item $[\hat{H},\hat{p}]\psi(x)=(\hat{H}\hat{p}-\hat{p}\hat{H})\psi(x)$
    \\Since $\hat{H}=\hat{K}+\hat{V},\hat{V}=0,\hat{K}=\frac{\hat{p}^2}{2m}$
    \\Then $[\hat{H},\hat{p}]\psi(x)=(\frac{\hat{p}^3}{2m}-\frac{\hat{p}^3}{2m})\psi(x)=0$
    \\\\Since $\hat{V}=0$, we can consider $\hat{H}=f(\hat{p})$ and $f$ is analytic function, then according to $[\hat{A},f(\hat{A})]=0$, we know the result is 0.
    \item For the momentum operator, the eigenfunction is $\phi_{p_x}(x)=Ce^{\frac{i}{\hbar}p_xx}$
    \\For Hamiltonian, since it's a free particle, $\hat{H}=\hat{K}=\frac{\hbar^2}{2m}\frac{\partial^2}{\partial x^2}$
    \\$\hat{H}\phi_{p_x}(x)=-\frac{\hbar^2}{2m}\frac{\partial^2\phi_{p_x}(x)}{\partial x^2}=\frac{C}{2m}p_x^2\phi_{p_x}(x)$
    \\\\Hence any eigenfunction of the 1D momentum operator is also an eigenfunction of the Hamiltonian of a free particle.
    \\\\Conversely, for the eigenfunction of $\hat{H}$ is $\psi(x)=A\cos kx+B\sin kx$ where $k=\sqrt{\frac{2mE}{\hbar}}$
    \\$\hat{p}\psi(x)=-i\hbar\frac{\partial(A\cos kx+B\sin kx)}{\partial x}=ik\hbar(A\sin kx-B\cos kx)$
    \\\\Hence the eigenfunction of the Hamiltonian of a free particle is not always the eigenfunction of the 1D momentum operator
    \item In this case, $\hat{H}=\hat{K}+\hat{V}=f(\hat{p})+\hat{V}$
    \\$[\hat{H},\hat{p}]\psi(x)=[f(\hat{p})+\hat{V},\hat{p}]\psi(x)=([f(\hat{p}),\hat{p}]+[\hat{V},\hat{p}])\psi(x)=[\hat{V},\hat{p}]\psi(x)$
    \\$[\hat{H},\hat{p}]\psi(x)=V(x)-i\hbar\frac{\partial\psi(x)}{\partial x}+i\hbar\frac{\partial V(x)\psi(x)}{\partial x}=-i\hbar V(x)\psi(x)'+[i\hbar V(x)\psi(x)'+i\hbar V(x)'\psi(x)]=i\hbar V(x)'\psi(x)$
    \\For two operators $\hat{A},\hat{B}$ where one is a function while another takes one order derivative, their commutator will take the derivative of the operator's function instead of the input function, because it's cancelled out. 
\end{enumerate}
\section{Problem 8}
\noindent According to the general uncertainty principle $\bigtriangleup_E\bigtriangleup_x\geq\frac{1}{2}|\braket{\Psi,i[\hat{H},\hat{x}]\Psi}|$
\\$i[\hat{H},\hat{x}]=i(\hat{H}\hat{x}-\hat{x}\hat{H})=i\{[-i\frac{\hbar^2}{2m}\frac{\partial^2}{\partial x^2}+V(x)]x-xV(x)+xi\frac{\hbar^2}{2m}\frac{\partial^2}{\partial x^2}\}=\frac{\hbar^2}{2m}\frac{\partial^2}{\partial x^2}x-x\frac{\hbar^2}{2m}\frac{\partial^2}{\partial x^2}$
\\$i[\hat{H},\hat{x}]=\frac{\hbar^2}{2m}(\frac{2\partial}{\partial x}+x\frac{\partial^2}{\partial x^2}-x\frac{\partial^2}{\partial x^2}=\frac{\hbar^2}{m}\frac{\partial}{\partial x}=i\frac{\hbar}{m}(-i\hbar\frac{\partial}{\partial x})=i\frac{\hbar}{m}\hat{p}$
\\Hence $\bigtriangleup_E\bigtriangleup_x\geq\frac{1}{2}|\braket{\Psi,i\frac{\hbar}{m}\hat{p}\Psi}|=\frac{\hbar}{2m}\braket{\hat{p}}$
\section{Problem 9}
\begin{enumerate}
    \item No, since the polarization direction of $A$ and $B$ are perpendicular to each other, the vertically polarized light can't penetrate through $B$
    \item \begin{enumerate}[(i)]
        \item According to Malus's Law, $I=I_0\cos^2\frac{\pi}{2}\cos^2\frac{\pi}{2}$, behind the $B$ we can get light with intensity $I=\frac{1}{4}I_0$
        \item With the increase of $\theta$, the light intensity behind $C$ is $I=I_0\cos^2\theta\sin^2\theta$, which is max when $\theta=\frac{\pi}{2}$ and minimum when $\theta=0$ 
    \end{enumerate} 
    According to our intuition, we should never get light behind $B$ because the polarization direction of $A$ and $B$ are perpendicular to each other, however, it fails when we put another polarizer $C$ between then when $\theta\neq0$
    \par In this guess, I guess the polarizer $C$ plays a role like measurement or observation for the polarization of the light when $\theta\neq0$, and it disturb the light's original polarization. Hence, Malus's law is an example showing how observation will cause the collapse of wave function    
\end{enumerate}
\section*{Python Code for Problem 1}
    \begin{verbatim}
        import numpy as np
        import matplotlib.pyplot as plt
        import math
        t0=1
        a=1e34
        h=1.05e-34
        m=9e-31
        p0=1e30
        t=30
        x=np.linspace(-100,100,10000)
        b=a*h*(1+t*2/(t0**2))**0.5
        psi=math.e**(-(x-p0*m*t)**2/(b**2))/(b*math.pi**0.5)
        title='The probability function at t=' + str(t)
        plt.title(title)
        plt.grid(1)
        plt.plot(x,psi)
        plt.show()
    \end{verbatim}
\end{document}