\documentclass[a4paper]{article}
\usepackage{amsmath}
\usepackage{amssymb}
\usepackage{braket}%量子力学符号
\usepackage{geometry}
\usepackage{enumerate}
\usepackage{natbib}
\usepackage{float}%稳定图片位置
\usepackage{graphicx,subfig}%画图
\usepackage{caption}
\usepackage[english]{babel}
\usepackage{indentfirst}%缩进
\usepackage{enumerate}%加序号
\usepackage{multirow}%合并行
\usepackage{hyperref}
\hypersetup{hypertex=true, colorlinks=true, linkcolor=black, anchorcolor=black, citecolor=black}
\title{\Large \textbf{VP390 Problem Set 3}\\
\author{\textbf{Pan, Chongdan ID:516370910121}\\
}
}
\begin{document}
\maketitle
\section{Problem 1}
\begin{enumerate}[(a)]
    \item $|\Psi(x,t)|^2=\Psi(x,t)\Psi^*(x,t)=Ce^{-\frac{i}{\hbar}Et}(x^2-a^2)Ce^{\frac{i}{\hbar}Et}(x^2-a^2)=C^2(x^2-a^2)^2$
    \\ $|\Psi(x,t)|^2=\Phi(x)$ So $\Psi$ represents a stationary state.
    \item $\int_{-a}^a|\Psi(x,t)|^2\mathrm{d}x=\int_{-a}^a C^2(x^2-a^2)^2\mathrm{d}x=C^2(\frac{2a^5}{5}-\frac{4a^5}{3}+2a^5)=C^2\frac{16a^5}{15}=1$
    \\$C^2a^5=\frac{15}{16}$
    \item $\int_{-a/2}^{a/2} C^2(x^2-a^2)^2\mathrm{d}x=C^2(\frac{a^5}{80}-\frac{a^5}{6}+a^5)=\frac{203}{256}\approx0.793$
    \item Since $\Phi(x)$ is symmetric about $x$-axis, the probability is 0.5
    \item $\braket{x}_\Psi=\int_{-\infty}^\infty x|\Psi(x,t)|^2\mathrm{d}x=\int_{-a}^a C^2x(x^2-a^2)^2\mathrm{d}x=0$
    So it's time independent
    \item $\bigtriangleup_x=\sqrt{\braket{x^2}_\Psi-\braket{x}_\Psi^2}=C\sqrt{\int_{-a}^a x^2(x^-a^2)^2\mathrm{d}x}=C\sqrt{\frac{30a^7}{105}-\frac{84a^7}{105}+\frac{70a^7}{105}}=\sqrt{\frac{a^2}{7}}$
\end{enumerate}
\section{Problem 2}
\begin{enumerate}[(a)]
    \item $E_1=13.6$eV, $a_0=\frac{4\pi\epsilon_0\hbar^2}{me^2}\approx5.29\times10^{-2}$nm
    \\$\int_{-\pi/2}^{\pi/2}\int_0^{2\pi}\int_0^\infty\Psi_{100}(r,\varphi,\theta,t)|^2\mathrm{d}r\mathrm{d}\theta\mathrm{d}\varphi=C^2\int_{-\pi/2}^{\pi/2}\int_0^{2\pi}\int_0^\infty r^2\sin\varphi\exp(-\frac{2r}{a_0})\mathrm{d}r\mathrm{d}\theta\mathrm{d}\varphi=2\times2\pi C^2\frac{a^3}{4}$
    \\$C=\sqrt{\frac{1}{\alpha_0^3\pi}}=1.467\times10^{15}$
    \item $C^2\int_{-\pi/2}^{\pi/2}\int_0^{2\pi}\int_0^{a_0} r^2\sin\varphi\exp(-\frac{2r}{a_0})\mathrm{d}r\mathrm{d}\theta\mathrm{d}\varphi=\pi C^2(a^3-5a^3e^{-2})=1-5e^{-2}$
    \item The probability for the electron radius is $r$ can be expressed :
    \\$C^2\int_{-\pi/2}^{\pi/2}\int_0^{2\pi} r^2\sin\varphi\exp(-\frac{2r}{a_0})\mathrm{d}\theta\mathrm{d}\varphi=4\pi C^2r^2\exp(-\frac{2r}{a_0})$
    \\$\frac{\mathrm{d}r^2\exp(-\frac{2r}{a_0})}{\mathrm{d}r}=2r\exp(-\frac{2r}{a_0})-\frac{2r^2}{a_0}\exp(-\frac{2r}{a_0})=0$
    \\$r_1=0, r_2=a_0, r_3=\infty$, since $r_1$ and $r_3$ can lead to the minium value of $r^2\exp(-\frac{2r}{a_0})$, only $r=\alpha_0$ has the most probability value to find the electron
    \item $r_a=\int_{-\pi/2}^{\pi/2}\int_0^{2\pi}\int_0^\infty r\Psi_{100}(r,\varphi,\theta,t)|^2\mathrm{d}r\mathrm{d}\theta\mathrm{d}\varphi=C^2\int_{-\pi/2}^{\pi/2}\int_0^{2\pi}\int_0^\infty r^3\sin\varphi\exp(-\frac{2r}{a_0})\mathrm{d}r\mathrm{d}\theta\mathrm{d}\varphi$
    \\$=4\pi C^2\frac{3a_0^4}{8}=\frac{3a_0}{2}$
    \\$\bigtriangleup_r=\sqrt{\braket{r^2}_\Psi-r_a^2}$
    \\$\braket{r^2}=4\pi C^2\int_0^\infty r^4\exp(-\frac{2r}{a_0})\mathrm{d}r=3\pi C^2 a_0^5=3a_0^2$
    \\$\bigtriangleup_r=\sqrt{3a_0^2-\frac{9a_0^2}{4}}=\frac{\sqrt{3}a_0}{2}$
    \\So the average value of $r$ is $\frac{3a_0}{2}$, its standard deviation is $\frac{\sqrt{3}a_0}{2}$
\end{enumerate}
\section{Problem 3}
\begin{enumerate}[(a)]
    \item $\braket{f,g}=\int_{-\infty}^\infty f^*(u)g(u)\mathrm{d}u=\int_{-\infty}^\infty (g^*(u))^*f^*(u)\mathrm{d}u=(\int_{-\infty}^\infty g^*(u)f(u)\mathrm{d}u)^*=\braket{g,f}^*$
    \item $\braket{f,\alpha g}=\int_{-\infty}^\infty f^*(u)\alpha g(u)\mathrm{d}u=\alpha\int_{-\infty}^\infty f^*(u)g(u)\mathrm{d}u=\alpha\braket{f,g}$
    \item $\braket{\alpha f,g}=\braket{g,\alpha f}^*=\alpha^*\braket{g,f}^*=\alpha^*\braket{f,g}$
\end{enumerate}
\section{Problem 4}
    \begin{enumerate}
        \item $\int_{-\infty}^\infty|\psi|^2\mathrm{d}x=\int_0^L\frac{2}{L}\sin^2(\frac{n\pi x}{L})\mathrm{d}x=\frac{2}{L}(\frac{L}{2}-\frac{L\sin 2n\pi}{4n\pi})=1$
        \\So it is normalized
        \item $\braket{\psi_n,\psi_m}=\int_0^L\frac{2}{L}\sin\frac{n\pi x}{L}\sin\frac{m\pi x}{L}\mathrm{d}x=\frac{1}{L}\int_0^L\cos\frac{(n-m)\pi x}{L}-\cos\frac{(n+m)\pi x}{L}\mathrm{d}x$
        \\$=\frac{1}{L}[\frac{L\sin(n-m)\pi}{(n-m)\pi}-\frac{L\sin(n+m)\pi}{(n+m)\pi}]=0$
        \item $\braket{i\psi_1,-3\psi_1+2i\psi_2-\psi_3}=3i\braket{\psi_1,\psi_1}-2\braket{\psi_1,\psi_2}-i\braket{\psi_1,\psi_3}=3i$
        \item Assume $x=x_1\psi_1+x_2\psi_2,y=y_1\psi_1+y_2\psi_2$ where
        \\$x_1^2+x_2^2=1,y_1^2+y_2^2=1,x_1y_1+x_2y_2=0$
        \\$x=\frac{\sqrt{2}}{2}\psi_1-\frac{\sqrt{2}}{2}\psi_2,y=\frac{\sqrt{2}}{2}\psi_1+\frac{\sqrt{2}}{2}\psi_2$
    \end{enumerate}
\section{Problem 5}
    \begin{enumerate}
        \item $\braket{\psi,\psi}=\frac{1}{2}[\braket{e^{-\frac{i}{\hbar}E_1t}\psi_1(x),e^{-\frac{i}{\hbar}E_1t}\psi_1(x)}+\braket{e^{-\frac{i}{\hbar}E_2t}\psi_2(x),e^{-\frac{i}{\hbar}E_2t}\psi_2(x)}]$
        \\$\frac{1}{2}(\braket{\psi_1,\psi_1}+\braket{\psi_2,\psi_2})=1$
        \\So it's normalized
        \item $\Psi(x,t)^*\Psi(x,t)=\frac{1}{2}(e^{\frac{i}{\hbar}E_1t}\psi_1^*(x)+e^{\frac{i}{\hbar}E_2t}\psi_2^*(x))(e^{-\frac{i}{\hbar}E_1t}\psi_1(x)+e^{-\frac{i}{\hbar}E_2t}\psi_2(x))$
        \\$=\frac{1}{2}(\psi_1^*(x)\psi_1(x)+\psi_2^*(x)\psi_2(x))+\frac{1}{2}(e^{\frac{i}{\hbar}(E_1-E_2)t}\psi_1^*(x)\psi_2(x)+e^{\frac{i}{\hbar}(E_2-E_1)t}\psi_1(x)\psi_2^*(x))$
        \\So the $\Psi$ only describe a particle in a stationary state when $E_1=E_2$
        \item $\Psi(x,t)=\frac{1}{\sqrt{2}}(\cos\frac{E_1t}{\hbar}\psi_1(x)-i\sin\frac{E_1t}{\hbar}\psi_1(x)+\cos\frac{4E_1t}{\hbar}\psi_2(x)-i\sin\frac{4E_1t}{\hbar}\psi_2(x))$
        \\$T=\frac{2\pi}{\omega}=\frac{h}{E_1}$
        \\if $|\Psi(x,t+T)|^2=|\Psi(x,t)|^2=$ then 
        \\$\frac{1}{2}(e^{-\frac{i}{\hbar}3t}\psi_1^*(x)\psi_2(x)+e^{\frac{i}{\hbar}3t}\psi_1(x)\psi_2^*(x)=\frac{1}{2}(e^{-\frac{i}{\hbar}3E_1(t+T)}\psi_1^*(x)\psi_2(x)+e^{\frac{i}{\hbar}3E_1(t+T)}\psi_1(x)\psi_2^*(x)$
        \\$T=\frac{2\pi}{\omega}=\frac{h}{3E_1}$
        
    \end{enumerate}
\end{document}