\documentclass[12pt]{article}
\usepackage{amsmath}
\usepackage{amssymb}
\usepackage{geometry}
\usepackage{enumerate}
\usepackage{natbib}
\usepackage{float}%稳定图片位置
\usepackage{graphicx}%画图
\usepackage[english]{babel}
\usepackage{a4wide}
\usepackage{indentfirst}%缩进
\usepackage{enumerate}%加序号
\usepackage{multirow}%合并行
\title{\large UM-SJTU JOINT INSTITUTE\\Advanced Lasers and Optics Laboratory\\(VE438)\\\ \\\ \\\ \\\ \\\ \\\ \\\ \\\ \\\ \\\ \\\
Pre Lab Assignment\\\ \\\ LAB 6\\\ Acoustic-Optic Modulator \\\ \\\ \\\ \\\ \\\ }
\author{Name: Pan Chongdan \\ID: 516370910121}
\date{Date: \today}

\begin{document}
\maketitle
\newpage
\section{Answers for Pre Lab Questions}
\subsection{Question 1}
AOM is equivalent to a granting with period $d=\lambda_{sound}$, so that the relationship can be expressed as $\lambda_{sound}(\sin\theta\pm\sin i)=m\lambda$ where $i$ is the incident angle and $\theta$ is the diffraction angle, $m$ is the order, $\lambda$ is the wavelength of light and $\lambda_{sound}$ is the wavelength of sound. According to Bragg condition, we can obtain the biggest diffraction angle with $\sin\theta=m\frac{\lambda}{\lambda_{sound}}$. If we apply the frequency the sound, then the relation becomes $\frac{v_{sound}}{f_{sound}}(\sin\theta\pm\sin i)=m\lambda$, where $v_{sound}$ is the velocity of sound and $f_{sound}$ is sound's frequency.
\par When the light passes through the sound, its frequency $f'=f_{light}+f_{sound}$, then it's wavelength becomes $\lambda'=\frac{c}{f_{light}+mf_{sound}}$
\subsection{Question 2}
The frequency shift is equal to $m\omega$ where $m=0, \pm1, \pm2$ and the wavelength becomes $\frac{c\lambda}{c+m\omega\lambda}$, so the coherent length becomes $L=\frac{\lambda^2}{n(\lambda-\frac{c\lambda}{c+m\omega\lambda})}$ where $n$ is the refractive index.
\end{document}