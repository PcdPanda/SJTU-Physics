\documentclass[12pt]{article}
\usepackage{amsmath}
\usepackage{amssymb}
\usepackage{geometry}
\usepackage{enumerate}
\usepackage{natbib}
\usepackage{float}%稳定图片位置
\usepackage{graphicx}%画图
\usepackage[english]{babel}
\usepackage{a4wide}
\usepackage{indentfirst}%缩进
\usepackage{enumerate}%加序号
\usepackage{multirow}%合并行
\title{\large UM-SJTU JOINT INSTITUTE\\Advanced Lasers and Optics Laboratory\\(VE438)\\\ \\\ \\\ \\\ \\\ \\\ \\\ \\\ \\\ \\\ \\\
Post Lab Assignment\\\ \\\ LAB 5\\\ Spectrometer \\\ \\\ \\\ \\\ \\\ }
\author{Name: Pan Chongdan \\ID: 516370910121}
\date{Date: \today}

\begin{document}
\maketitle
\newpage
\section{Answers for Post Lab Questions}
\subsection{Question 1}
We estimate the period by the equation $d=\frac{m\lambda}{\sin\theta - \sin d}$ we calculate the difference because the incident light and diffraction light are at two sides of the normal line. Then we use the $\lambda$ of different color and calculate the angle for them respectively. The detail has been handed in during the lab. Through calculation of green light, our result is about $\frac{500}{\sin{26^o}-\sin{6^o}}=1497nm$, which is the period.
\subsection{Question 2}
According to $d=\frac{m\lambda}{\sin\theta - \sin d}$, when it's zero order, $d=\arcsin\theta=45^o$, when it's first order, $d=\arcsin(\sin\theta-\frac{\lambda}{d})=\arcsin(\frac{\sqrt{2}}{2}-\frac{500}{1497})=21.9^o$. If we use the zero-order diffraction, then the diffraction angle is equal to the incident angle so that the light can't be split out while for first order, the diffraction angle will be different for light of different wavelength. As a result, light of different color will split out when first order diffraction is used.
\subsection{Question 3}
According to the equation, when d doesn't change, since visible light has smaller wavelength, so the mid-IR spread more widely.
\end{document}